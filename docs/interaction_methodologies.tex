% Options for packages loaded elsewhere
\PassOptionsToPackage{unicode}{hyperref}
\PassOptionsToPackage{hyphens}{url}
\PassOptionsToPackage{dvipsnames,svgnames,x11names}{xcolor}
%
\documentclass[
  letterpaper,
  DIV=11,
  numbers=noendperiod]{scrartcl}

\usepackage{amsmath,amssymb}
\usepackage{iftex}
\ifPDFTeX
  \usepackage[T1]{fontenc}
  \usepackage[utf8]{inputenc}
  \usepackage{textcomp} % provide euro and other symbols
\else % if luatex or xetex
  \usepackage{unicode-math}
  \defaultfontfeatures{Scale=MatchLowercase}
  \defaultfontfeatures[\rmfamily]{Ligatures=TeX,Scale=1}
\fi
\usepackage{lmodern}
\ifPDFTeX\else  
    % xetex/luatex font selection
\fi
% Use upquote if available, for straight quotes in verbatim environments
\IfFileExists{upquote.sty}{\usepackage{upquote}}{}
\IfFileExists{microtype.sty}{% use microtype if available
  \usepackage[]{microtype}
  \UseMicrotypeSet[protrusion]{basicmath} % disable protrusion for tt fonts
}{}
\makeatletter
\@ifundefined{KOMAClassName}{% if non-KOMA class
  \IfFileExists{parskip.sty}{%
    \usepackage{parskip}
  }{% else
    \setlength{\parindent}{0pt}
    \setlength{\parskip}{6pt plus 2pt minus 1pt}}
}{% if KOMA class
  \KOMAoptions{parskip=half}}
\makeatother
\usepackage{xcolor}
\setlength{\emergencystretch}{3em} % prevent overfull lines
\setcounter{secnumdepth}{-\maxdimen} % remove section numbering
% Make \paragraph and \subparagraph free-standing
\makeatletter
\ifx\paragraph\undefined\else
  \let\oldparagraph\paragraph
  \renewcommand{\paragraph}{
    \@ifstar
      \xxxParagraphStar
      \xxxParagraphNoStar
  }
  \newcommand{\xxxParagraphStar}[1]{\oldparagraph*{#1}\mbox{}}
  \newcommand{\xxxParagraphNoStar}[1]{\oldparagraph{#1}\mbox{}}
\fi
\ifx\subparagraph\undefined\else
  \let\oldsubparagraph\subparagraph
  \renewcommand{\subparagraph}{
    \@ifstar
      \xxxSubParagraphStar
      \xxxSubParagraphNoStar
  }
  \newcommand{\xxxSubParagraphStar}[1]{\oldsubparagraph*{#1}\mbox{}}
  \newcommand{\xxxSubParagraphNoStar}[1]{\oldsubparagraph{#1}\mbox{}}
\fi
\makeatother


\providecommand{\tightlist}{%
  \setlength{\itemsep}{0pt}\setlength{\parskip}{0pt}}\usepackage{longtable,booktabs,array}
\usepackage{calc} % for calculating minipage widths
% Correct order of tables after \paragraph or \subparagraph
\usepackage{etoolbox}
\makeatletter
\patchcmd\longtable{\par}{\if@noskipsec\mbox{}\fi\par}{}{}
\makeatother
% Allow footnotes in longtable head/foot
\IfFileExists{footnotehyper.sty}{\usepackage{footnotehyper}}{\usepackage{footnote}}
\makesavenoteenv{longtable}
\usepackage{graphicx}
\makeatletter
\newsavebox\pandoc@box
\newcommand*\pandocbounded[1]{% scales image to fit in text height/width
  \sbox\pandoc@box{#1}%
  \Gscale@div\@tempa{\textheight}{\dimexpr\ht\pandoc@box+\dp\pandoc@box\relax}%
  \Gscale@div\@tempb{\linewidth}{\wd\pandoc@box}%
  \ifdim\@tempb\p@<\@tempa\p@\let\@tempa\@tempb\fi% select the smaller of both
  \ifdim\@tempa\p@<\p@\scalebox{\@tempa}{\usebox\pandoc@box}%
  \else\usebox{\pandoc@box}%
  \fi%
}
% Set default figure placement to htbp
\def\fps@figure{htbp}
\makeatother
% definitions for citeproc citations
\NewDocumentCommand\citeproctext{}{}
\NewDocumentCommand\citeproc{mm}{%
  \begingroup\def\citeproctext{#2}\cite{#1}\endgroup}
\makeatletter
 % allow citations to break across lines
 \let\@cite@ofmt\@firstofone
 % avoid brackets around text for \cite:
 \def\@biblabel#1{}
 \def\@cite#1#2{{#1\if@tempswa , #2\fi}}
\makeatother
\newlength{\cslhangindent}
\setlength{\cslhangindent}{1.5em}
\newlength{\csllabelwidth}
\setlength{\csllabelwidth}{3em}
\newenvironment{CSLReferences}[2] % #1 hanging-indent, #2 entry-spacing
 {\begin{list}{}{%
  \setlength{\itemindent}{0pt}
  \setlength{\leftmargin}{0pt}
  \setlength{\parsep}{0pt}
  % turn on hanging indent if param 1 is 1
  \ifodd #1
   \setlength{\leftmargin}{\cslhangindent}
   \setlength{\itemindent}{-1\cslhangindent}
  \fi
  % set entry spacing
  \setlength{\itemsep}{#2\baselineskip}}}
 {\end{list}}
\usepackage{calc}
\newcommand{\CSLBlock}[1]{\hfill\break\parbox[t]{\linewidth}{\strut\ignorespaces#1\strut}}
\newcommand{\CSLLeftMargin}[1]{\parbox[t]{\csllabelwidth}{\strut#1\strut}}
\newcommand{\CSLRightInline}[1]{\parbox[t]{\linewidth - \csllabelwidth}{\strut#1\strut}}
\newcommand{\CSLIndent}[1]{\hspace{\cslhangindent}#1}

\KOMAoption{captions}{tableheading}
\makeatletter
\@ifpackageloaded{caption}{}{\usepackage{caption}}
\AtBeginDocument{%
\ifdefined\contentsname
  \renewcommand*\contentsname{Table of contents}
\else
  \newcommand\contentsname{Table of contents}
\fi
\ifdefined\listfigurename
  \renewcommand*\listfigurename{List of Figures}
\else
  \newcommand\listfigurename{List of Figures}
\fi
\ifdefined\listtablename
  \renewcommand*\listtablename{List of Tables}
\else
  \newcommand\listtablename{List of Tables}
\fi
\ifdefined\figurename
  \renewcommand*\figurename{Figure}
\else
  \newcommand\figurename{Figure}
\fi
\ifdefined\tablename
  \renewcommand*\tablename{Table}
\else
  \newcommand\tablename{Table}
\fi
}
\@ifpackageloaded{float}{}{\usepackage{float}}
\floatstyle{ruled}
\@ifundefined{c@chapter}{\newfloat{codelisting}{h}{lop}}{\newfloat{codelisting}{h}{lop}[chapter]}
\floatname{codelisting}{Listing}
\newcommand*\listoflistings{\listof{codelisting}{List of Listings}}
\makeatother
\makeatletter
\makeatother
\makeatletter
\@ifpackageloaded{caption}{}{\usepackage{caption}}
\@ifpackageloaded{subcaption}{}{\usepackage{subcaption}}
\makeatother

\usepackage{bookmark}

\IfFileExists{xurl.sty}{\usepackage{xurl}}{} % add URL line breaks if available
\urlstyle{same} % disable monospaced font for URLs
\hypersetup{
  pdftitle={interaction\_methodologies},
  colorlinks=true,
  linkcolor={blue},
  filecolor={Maroon},
  citecolor={Blue},
  urlcolor={Blue},
  pdfcreator={LaTeX via pandoc}}


\title{interaction\_methodologies}
\author{}
\date{}

\begin{document}
\maketitle


\textbf{Results}

\emph{Interaction plant diversity}

Each survey type detected a different number of plant taxa within the B.
pascuorum interaction network (Figure 1). Flower count surveys
identified a total of 117 flowering plant genera available for
interactions within our transects across the sampling season.
Interaction transect surveys detected B. pascuorum interactions with 27
genera, all of which were also detected by other methodologies. Gut
content metabarcoding detected interactions with 132 total plant genera,
including 34 taxa uniquely identified by this methodology. Flower counts
detected 48\% of the taxa detected by gut metabarcoding, and pollen
metabarcoding shared 62\% of the taxa detected in gut samples.
Corbicular pollen metabarcoding detected 123 plant genera, with 31~ taxa
uniquely detected by this methodology. 56\% of of the taxa detected by
pollen metabarcoding were represented within the flower count survey
taxa, and 67\% of the pollen metabarcoding taxa were also identified by
gut metabarcoding.~

While, by design, flower counts and interaction transects detected only
taxa from the flowering entomophilous community, both metabarcoding
methodologies detected taxa from the anemophilous community as well.
Anemophilous plant taxa represented 18\% of the total identified plant
taxa between the two methodologies, including mainly grasses, but also
several tree genera, many detected across multiple samples.~

Taxonomic diversity captured by each methodology across periods revealed
an overall increase in interaction taxa over the sampling season for
each methodology (Figure 2). Divided by periods, detection patterns
followed the general results in that metabarcoding methodologies
consistently detected more taxa than interaction transects, with gut
metabarcoding detecting more taxa than other interaction methodologies
in each period. Flower counts detected more taxa than observed in all
other sampling methodologies for three of the six periods.

Bumblebees (Bombus spp.) are excellent model pollinators for numerous
reasons. On top of the importance of bees in pollen transport relative
to other pollinators (Alarcón 2010), bumblebees specifically offer the
advantage of abundance and ease of observation in the wild, common use
in agricultural settings (Velthuis and Doorn 2006) and climate
dependence (Sirois-Delisle and Kerr 2018) (Maebe et al. 2021). Given the
quantity of existing research on bumblebees, their environmental and
nutritional needs are already broadly defined. Bees generally need a
combination of pollen and nectar to meet their nutritional needs (Vaudo
2015). Pollen is the main source of protein and fats in the bee diet,
and the microbial content of pollen is an important source of nutrition
during the larval period (Steffan et al. 2019), (Dharampal et al. 2019).
A diversity of pollen sources from floral species promotes healthy
individual physiology (Di Pasquale et al. 2016) and for this reason it
is generally understood that plant diversity supports bee populations.
In addition to affecting biodiversity available to bees for forage,
landscape use has an effect on the pollinator internal microbiome and
individual health (Jones et al. 2018). While these relationships
indicate that a greater diversity of potential plant-pollinator
interactions in the environment promotes bee health, it is also
important to understand the finer details of interactions, including
which are most important to pollinators, and in what spatiotemporal
contexts.~

We evaluated the independent and complementary abilities of three
pollinator-perspective based methodologies for observation of \emph{B.
pascuorum} and plant interactions, contextualized by surveys of floral
resource availability. We compared each methodology in terms of their
abilities to capture diversity in interaction taxa and their data
outputs in relation to sampling effort. In this comparison,
metabarcoding methodologies far outperformed interaction transects in
terms of detected taxonomic diversity (Figure 1). This advantage was
maintained across the sampling season, as interaction networks generally
expanded (Figure 2). Considering this, and the effort involved in data
collection for both methodologies, metabarcoding also outperformed
interactions in terms of efficiency. Interaction transect taxa
inherently represented a subset of the taxa observed by flower counts,
with cumulative interaction observations including only 23\% of the taxa
identified across the season by flower counts. Metabarcoding
methodologies, however, were not limited to detection of interactions
included within the spatial constraint, observation periods, nor the
floral diversity of the transects. Both metabarcoding methodologies
observed greater total interaction taxa than the floral diversity
surveys. During three consecutive sampling periods from May to mid-July,
however, more taxa were present within the floral diversity of the
transects than metabarcoding detected in \emph{B. pascuorum}
interactions (Figure 2). Finally, comparing the two metabarcoding
methodologies, gut metabarcoding outperformed corbicular pollen
metabarcoding cumulatively and in each sampling period in terms of
taxonomic diversity captured. Comparing the sampling effort of both
methodologies, gut metabarcoding also presented an easier methodology,
given that the number of samples was equal to the number of individuals
captured, while not all captured individuals carried sufficient pollen
for genomic DNA extraction. These results suggested that in a raw
evaluation of capacity and demanded effort, gut metabarcoding provided
the best methodology for detecting interaction networks. However, the
seasonal complexities observed between floral resource availability and
interaction observations hint that the communities included in these
networks are also an important factor in choosing methodologies for
future studies.

We also compared community composition as detected by each methodology,
analyzing the difference in the plant taxa included in the interaction
network of \emph{B. pascuorum} as detected by the each methodology per
sampling day. We observed the expected ordination overlap (Figure 3)
between flower counts and interaction transects inherent in our study
design, although statistically the communities of the two methodologies
differed (Table 1). We attributed the statistical difference to the
large difference in number of taxa observed between the two transect
methodologies. Community composition analysis found no significant
differences between the two metabarcoding methodologies, but both
overlapped minimally with the flower count community and showed a
statistical difference to the results of this methodology. This
reflected an underlying dynamic of the \emph{B. pascuorum} interaction
network structure, first signaled by the differences in taxonomic
diversity observed by these methodologies across periods (Figure 2). The
core difference between the metabarcoding methodologies and flower
counts was that the latter ultimately did not represent a methodology
for observing interactions, but rather a survey of entomophilous taxa
that could be included in interactions. As with interaction transects,
flower counts were also limited in space, time, and taxonomy, while
metabarcoding methodologies were unrestricted by these boundaries.

Our results suggest that metabarcoding has a vast ability to expand our
existing understanding of pollinator interaction networks. Consistent
with previous comparisons between field and metabarcoding observation of
plant-pollinator interactions (Milla et al. 2022), (Baksay et al.
2022),(Smart et al. 2017), metabarcoding increased observed interaction
diversity by more than six-fold. Metabarcoding detected 94 total plant
taxa that were not included in the 117 identified by the floral
diversity surveys detected within our transects. This suggests that a
large number of taxa within the \emph{B. pascuorum} interaction network
lie either beyond the spatial extent of our transects or outside of
entomophilous community traditionally sampled in plant-pollinator
network surveys. Our reference database for metabarcode sequences
included all 68,420 ITS2 sequences, including all dicotyledons and
select additional plant sequences (K. Bell 2021), allowing metabarcoding
to identify taxa from beyond the entomophilous community. Our
metabarcoding data revealed a variety of anemophilous taxa within the
\emph{B. pascuorum} interaction network, including trees and shrubs (n =
9), grasses (n =15), and other herbaceous plants (n = 7), indicating a
greater diversity and spatial range of bumblebee foraging than
previously expected.~Indeed, our transects did not extend beyond grass
and shrub meadows, although other habitat types were adjacent or
partially integrated into to the transect areas, including forests and
wetlands.

Our observations of interactions with the anemophilous community support
other previously documented interaction observations (Selva et al. 2024)
(Milla et al. 2022) (Timberlake et al. 2024) (Vere et al. 2017) (Tanaka
et al. 2020) {[}Terrell and Batra (1984){]}(Pojar 1973), and have
especially intriguing implications in the context of gut metabarcoding.
Previous studies using external pollen metabarcoding have removed
anemophilous taxa, specifically wind-pollinated taxa from their analyses
(e.g.(Tanaka et al. 2020), (Pornon et al. 2017)). It is understandable
that in an external pollen metabarcoding context with the objective of
characterizing pollination interactions, that these sources of pollen
could be overwhelmingly represented in results. Our results suggest that
in studies intending to use external pollen loads as proxies for forage
networks, choosing to omit these taxa could be a large oversight. Gut
metabarcoding data are direct observations of consumed plant material,
and our finding of pollen or plant material consumed from anemophilous
species within or nearby our transects could support existing hypotheses
for pollinator forage adaptations in response to environmental changes.
For example, expanding forage diversity beyond the floral community and
across habitats is likely to explain how Bumblebees survive annual
``hunger gaps'' (Timberlake, Tew, and Memmott 2024) (Becher et al. 2024)
, when blooming floral species are limited (Timberlake et al. 2024).
Additionally, our observations are complementary to those that have
previously characterized bumblees as selective foragers, using an
ability to assess nutritional content of pollen to determine forage
choices (Leonhardt and Blüthgen 2012) based on quality, rather than
quantity. As the first study to employ metabarcoding as a means of
detecting plant taxa represented in pollinator gut contents, the success
of our methodology, especially as compared to alternative existing
methodologies, offers a promising direction for future plant-pollinator
interactions studies. This is especially evident when considering the
resolution of the metabarcoding data presented here, which have been
simplified for comparative purposes.

In comparing community composition across methodologies, we chose to
aggregate our metabarcoding data by sampling day. By doing this, we
ensured that sampling effort was balanced across methodologies. However,
this understated the resolution of the metabarcoding data. One of the
great advantages of applying metabarcoding to pollinator samples is the
ability to study interactions at the individual level {[}Gous et al.
(2019){]}(K. L. Bell et al. 2016). We compared the interaction data from
individual specimens with both between pollen and gut samples. Again,
more taxa were detected in gut samples (mean = 20 genera, sd = 10) than
in pollen samples (mean = 18 genera, sd = 7). With individual level
data, we were able to compare the number of taxa identified for each
specimen shared by both sample sources, which was on average only 15\%
of the detected taxa. This result indicated that both methodologies
could be detecting distinct communities related to the roles of the
uniquely detected taxa as sources of nutrition for different purposes.
forage for immediate or larval consumption (Vaudo 2015). The repeated
detection of certain taxa between both methodologies increased
confidence in these identifications, again demonstrating the
complementary potential of these methodologies. For example,
\emph{Vicia} was observed by both methodologies across all specimens
with both sample types in the first three periods. This adds to our
previous knowledge from interaction transects of the strong relationship
between \emph{Vicia pyrenaica} and \emph{B. pascuorum} (Artamendi,
unpublished data), demonstrating the value of field surveys in
validating laboratory based methodologies.

Our results suggest complementary roles between the interaction
methodologies. The statistical similarities between observed plant
communities suggests robustness between each of them, and the inherent
implications of the sample sources of each provide varied means of
interpreting different interactions. Interaction transects provide a
valuable field-based perspective, which is integral in ecology, and
especially important as it becomes increasingly laboratory based, and as
molecular tools mature in their reliability (maybe could cite
(Richardson et al. 2015) as a paper suggesting need for growth of
molecular techniques, but it's old). Interaction transects have the
previously observed disadvantage in sampling efficiency, suggesting that
a reduced approach, focused on validation of other methodologies, may be
the best way to integrate this methodology into future studies. As the
external pollen load is a direct observation of the pollen that can be
transferred between flowers, the taxa identified by pollen metabarcoding
represent those most likely to benefit from pollination services. Pollen
metabarcoding may also be a good starting point for identifying which
plants provide pollen with optimal macronutrients for larval growth,
given that external pollen will be returned to the nest to feed
developing larvae (Vaudo 2015). In a similar manner, gut metabarcoding
provides an important perspective on the nutritional needs of actively
foraging pollinators, identifying which taxa provide pollen as food for
supporting this activity (Li et al. 2025). Knowing which taxa are
actually ingested by pollinators is especially useful information for
identifying taxa that facilitate microbiota exchange and acquisition
during plant interactions (Cullen, Fetters, and Ashman 2021), including
parasite and disease transfer (Lignon et al. 2024). Given the different
information offered by each methodology,

Each interaction methodology, with the exception of gut content
metabarcoding, has been used previously as a standalone means for
characterizing plant-pollinator interaction networks (e.g. (Magrach et
al. 2023), (Devriese et al. 2024)). While our results suggest that a
single methodology approach may have limitations

As a result, the communities detected by metabarcoding were able to
include those of the transect methodologies, as well as parts of the
Gorbeia plant community beyond the transects, speaking to the
complementary nature of the data provided by each of the methodologies
together.

As standalone methodologies, each offers advantages and disadvantages in
terms of informative capacities, sensitivity, and sampling effort.
Interaction transects, fof

\phantomsection\label{refs}
\begin{CSLReferences}{1}{0}
\bibitem[\citeproctext]{ref-alarcuxf3n2010}
Alarcón, Ruben. 2010. {``Congruence Between Visitation and
Pollen-Transport Networks in a California Plant{\textendash}pollinator
Community.''} \emph{Oikos} 119 (1): 35--44.
\url{https://doi.org/10.1111/j.1600-0706.2009.17694.x}.

\bibitem[\citeproctext]{ref-baksay2022}
Baksay, Sandra, Christophe Andalo, Didier Galop, Monique Burrus,
Nathalie Escaravage, and André Pornon. 2022. {``Using Metabarcoding to
Investigate the Strength of Plant-Pollinator Interactions from Surveys
of Visits to DNA Sequences.''} \emph{Frontiers in Ecology and Evolution}
10 (March). \url{https://doi.org/10.3389/fevo.2022.735588}.

\bibitem[\citeproctext]{ref-becher2024}
Becher, Matthias A., Grace Twiston-Davies, Juliet L. Osborne, and Tonya
A. Lander. 2024. {``Resource Gaps Pose the Greatest Threat for
Bumblebees During the Colony Establishment Phase.''} \emph{Insect
Conservation and Diversity} 17 (4): 676--89.
\url{https://doi.org/10.1111/icad.12736}.

\bibitem[\citeproctext]{ref-bell2021}
Bell, Karen. 2021. {``ITS2 July 2021.''} figshare.
\url{https://doi.org/10.6084/M9.FIGSHARE.14936004.V1}.

\bibitem[\citeproctext]{ref-bell2016}
Bell, Karen L., Natasha de Vere, Alexander Keller, Rodney T. Richardson,
Annemarie Gous, Kevin S. Burgess, and Berry J. Brosi. 2016. {``Pollen
DNA Barcoding: Current Applications and Future Prospects.''}
\emph{Genome} 59 (9): 629--40.
\url{https://doi.org/10.1139/gen-2015-0200}.

\bibitem[\citeproctext]{ref-cullen2021}
Cullen, Nevin, Andrea Fetters, and Tia-Lynn Ashman. 2021. {``Integrating
Microbes into Pollination.''} \emph{Current Opinion in Insect Science}
44 (April): 48--54. \url{https://doi.org/10.1016/j.cois.2020.11.002}.

\bibitem[\citeproctext]{ref-devriese2024}
Devriese, Arne, Gerrit Peeters, Rein Brys, and Hans Jacquemyn. 2024.
{``The Impact of Extraction Method and Pollen Concentration on Community
Composition for Pollen Metabarcoding.''} \emph{Applications in Plant
Sciences} 12 (5): e11601. \url{https://doi.org/10.1002/aps3.11601}.

\bibitem[\citeproctext]{ref-dharampal2019}
Dharampal, Prarthana S., Caitlin Carlson, Cameron R. Currie, and Shawn
A. Steffan. 2019. {``Pollen-Borne Microbes Shape Bee Fitness.''}
\emph{Proceedings of the Royal Society B: Biological Sciences} 286
(1904): 20182894. \url{https://doi.org/10.1098/rspb.2018.2894}.

\bibitem[\citeproctext]{ref-dipasquale2016}
Di Pasquale, Garance, Cédric Alaux, Yves Le Conte, Jean-François Odoux,
Maryline Pioz, Bernard E. Vaissière, Luc P. Belzunces, and Axel
Decourtye. 2016. {``Variations in the Availability of Pollen Resources
Affect Honey Bee Health.''} \emph{PLOS ONE} 11 (9): e0162818.
\url{https://doi.org/10.1371/journal.pone.0162818}.

\bibitem[\citeproctext]{ref-gous2019}
Gous, Annemarie, Dirk Z. H. Swanevelder, Connal D. Eardley, and Sandi
Willows-Munro. 2019. {``Plant{\textendash}pollinator Interactions over
Time: Pollen Metabarcoding from Bees in a Historic Collection.''}
\emph{Evolutionary Applications} 12 (2): 187--97.
\url{https://doi.org/10.1111/eva.12707}.

\bibitem[\citeproctext]{ref-jones2018}
Jones, Julia C, Carmelo Fruciano, Falk Hildebrand, Hasan Al Toufalilia,
Nicholas J Balfour, Peer Bork, Philipp Engel, Francis LW Ratnieks, and
William OH Hughes. 2018. {``Gut Microbiota Composition Is Associated
with Environmental Landscape in Honey Bees.''} \emph{Ecology and
Evolution} 8 (1): 441--51. \url{https://doi.org/10.1002/ece3.3597}.

\bibitem[\citeproctext]{ref-leonhardt2012}
Leonhardt, Sara Diana, and Nico Blüthgen. 2012. {``The Same, but
Different: Pollen Foraging in Honeybee and Bumblebee Colonies.''}
\emph{Apidologie} 43 (4): 449--64.
\url{https://doi.org/10.1007/s13592-011-0112-y}.

\bibitem[\citeproctext]{ref-li2025}
Li, Yiran, Chengweiran Liu, Yiran Wang, Muhan Li, Shasha Zou, Xingyu Hu,
Zhiwei Chen, et al. 2025. {``Urban Wild Bee Well-Being Revealed by Gut
Metagenome Data: A Mason Bee Model.''} \emph{Insect Science} n/a (n/a).
\url{https://doi.org/10.1111/1744-7917.70051}.

\bibitem[\citeproctext]{ref-lignon2024}
Lignon, V. Aiko, Flore Mas, E. Eirian Jones, Clive Kaiser, and Manpreet
K. Dhami. 2024. {``The Floral Interface: A Playground for Interactions
Between Insect Pollinators, Microbes, and Plants.''} \emph{New Zealand
Journal of Zoology}, May, 1--20.
\url{https://doi.org/10.1080/03014223.2024.2353285}.

\bibitem[\citeproctext]{ref-maebe2021}
Maebe, Kevin, Alex F. Hart, Leon Marshall, Peter Vandamme, Nicolas J.
Vereecken, Denis Michez, and Guy Smagghe. 2021. {``Bumblebee Resilience
to Climate Change, Through Plastic and Adaptive Responses.''}
\emph{Global Change Biology} 27 (18): 4223--37.
\url{https://doi.org/10.1111/gcb.15751}.

\bibitem[\citeproctext]{ref-magrach2023}
Magrach, Ainhoa, Maddi Artamendi, Paula Dominguez Lapido, Clara Parejo,
and Encarnacion Rubio. 2023. {``Indirect Interactions Between
Pollinators Drive Interaction Rewiring Through Space.''}
\emph{Ecosphere} 14 (6): e4521. \url{https://doi.org/10.1002/ecs2.4521}.

\bibitem[\citeproctext]{ref-milla2022}
Milla, Liz, Alexander Schmidt-Lebuhn, Jessica Bovill, and Francisco
Encinas-Viso. 2022. {``Monitoring of Honey Bee Floral Resources with
Pollen DNA Metabarcoding as a Complementary Tool to Vegetation
Surveys.''} \emph{Ecological Solutions and Evidence} 3 (1): e12120.
\url{https://doi.org/10.1002/2688-8319.12120}.

\bibitem[\citeproctext]{ref-pojar1973}
Pojar, Jim. 1973. {``Pollination of Typically Anemophilous Salt Marsh
Plants by Bumble Bees, Bombus Terricola Occidentalis Grne.''} \emph{The
American Midland Naturalist} 89 (2): 448--51.
\url{https://doi.org/10.2307/2424049}.

\bibitem[\citeproctext]{ref-pornon2017}
Pornon, André, Christophe Andalo, Monique Burrus, and Nathalie
Escaravage. 2017. {``DNA Metabarcoding Data Unveils Invisible
Pollination Networks.''} \emph{Scientific Reports} 7 (1): 16828.
\url{https://doi.org/10.1038/s41598-017-16785-5}.

\bibitem[\citeproctext]{ref-richardson2015}
Richardson, Rodney T., Chia-Hua Lin, Juan O. Quijia, Natalia S. Riusech,
Karen Goodell, and Reed M. Johnson. 2015. {``Rank-Based Characterization
of Pollen Assemblages Collected by Honey Bees Using a Multi-Locus
Metabarcoding Approach.''} \emph{Applications in Plant Sciences} 3 (11):
1500043. \url{https://doi.org/10.3732/apps.1500043}.

\bibitem[\citeproctext]{ref-selva2024}
Selva, Simonetta, Marco Moretti, Fabian Ruedenauer, Alexander Keller,
Bertrand Fournier, Sara D. Leonhardt, Helen A. Eggenberger, and Joan
Casanelles Abella. 2024. {``Urban Bumblebees Diversify Their Foraging
Strategy to Maintain Nutrient Intake,''} October.
\url{https://ecoevorxiv.org/repository/view/7812/}.

\bibitem[\citeproctext]{ref-sirois-delisle2018}
Sirois-Delisle, Catherine, and Jeremy T. Kerr. 2018. {``Climate
Change-Driven Range Losses Among Bumblebee Species Are Poised to
Accelerate.''} \emph{Scientific Reports} 8 (1): 14464.
\url{https://doi.org/10.1038/s41598-018-32665-y}.

\bibitem[\citeproctext]{ref-smart2017}
Smart, M. D., R. S. Cornman, D. D. Iwanowicz, M. McDermott-Kubeczko, J.
S. Pettis, M. S. Spivak, and C. R. V. Otto. 2017. {``A Comparison of
Honey Bee-Collected Pollen from Working Agricultural Lands Using Light
Microscopy and ITS Metabarcoding.''} \emph{Environmental Entomology} 46
(1): 38--49. \url{https://doi.org/10.1093/ee/nvw159}.

\bibitem[\citeproctext]{ref-steffan2019}
Steffan, Shawn A., Prarthana S. Dharampal, Bryan N. Danforth, Hannah R.
Gaines-Day, Yuko Takizawa, and Yoshito Chikaraishi. 2019. {``Omnivory in
Bees: Elevated Trophic Positions Among All Major Bee Families.''}
\emph{The American Naturalist} 194 (3): 414--21.
\url{https://doi.org/10.1086/704281}.

\bibitem[\citeproctext]{ref-tanaka2020}
Tanaka, Keisuke, Akinobu Nozaki, Hazuki Nakadai, Yuh Shiwa, and Mariko
Shimizu-Kadota. 2020. {``Using Pollen DNA Metabarcoding to Profile
Nectar Sources of Urban Beekeeping in K{ō}t{ō}-Ku, Tokyo.''} \emph{BMC
Research Notes} 13 (1): 515.
\url{https://doi.org/10.1186/s13104-020-05361-2}.

\bibitem[\citeproctext]{ref-terrell1984}
Terrell, Edward E., and Suzanne W. T. Batra. 1984. {``Insects Collect
Pollen of Eastern Wildrice, Zizania Aquatica (Poaceae).''}
\emph{Castanea} 49 (1): 31--34.
\url{https://www.jstor.org/stable/4033059}.

\bibitem[\citeproctext]{ref-timberlake2024b}
Timberlake, T. P., N. E. Tew, and J. Memmott. 2024. {``Gardens Reduce
Seasonal Hunger Gaps for Farmland Pollinators.''} \emph{Proceedings of
the Royal Society B: Biological Sciences} 291 (2033).
\url{https://doi.org/10.1098/rspb.2024.1523}.

\bibitem[\citeproctext]{ref-timberlake2024}
Timberlake, T. P., N. de Vere, L. E. Jones, I. P. Vaughan, M. Baude, and
J. Memmott. 2024. {``Ten-a-Day: Bumblebee Pollen Loads Reveal High
Consistency in Foraging Breadth Among Species, Sites and Seasons.''}
\emph{Ecological Solutions and Evidence} 5 (3): e12360.
\url{https://doi.org/10.1002/2688-8319.12360}.

\bibitem[\citeproctext]{ref-vaudo2015}
Vaudo, Anthony D. 2015. {``Bee Nutrition and Floral Resource
Restoration.''} \emph{Current Opinion in Insect Science} 10 (August):
133--41. \url{https://doi.org/10.1016/j.cois.2015.05.008}.

\bibitem[\citeproctext]{ref-velthuis2006}
Velthuis, Hayo H. W., and Adriaan van Doorn. 2006. {``A Century of
Advances in Bumblebee Domestication and the Economic and Environmental
Aspects of Its Commercialization for Pollination.''} \emph{Apidologie}
37 (4): 421--51. \url{https://doi.org/10.1051/apido:2006019}.

\bibitem[\citeproctext]{ref-devere2017}
Vere, Natasha de, Laura E. Jones, Tegan Gilmore, Jake Moscrop, Abigail
Lowe, Dan Smith, Matthew J. Hegarty, Simon Creer, and Col R. Ford. 2017.
{``Using DNA Metabarcoding to Investigate Honey Bee Foraging Reveals
Limited Flower Use Despite High Floral Availability.''} \emph{Scientific
Reports} 7 (1): 42838. \url{https://doi.org/10.1038/srep42838}.

\end{CSLReferences}




\end{document}
